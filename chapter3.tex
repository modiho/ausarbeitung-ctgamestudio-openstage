\chapter{Ansatz}

Im Storymodus des ctGameStudios lernen Schüler grundlegende Konzepte der Programmierung wie
Schleifen, Verzweigungen, Ereignisse, Prozeduren und Funktionen kennen. In einer Erweiterung des
Spiels sollen sie nun die Möglichkeit haben, die erlenten Fähigkeiten anzuwenden. Dabei sollte 

\section{Ein Lernszenario}

Siehe Folien. Kontext: Informatikunterricht.

1. Schüler spielen Storymodus um Grundlagen zu lernen,
2. Schüler spielen Strategie-Traningsmodus um Erlentes anzuwenden und zu verbessern,
3. Lehrer erstellt Turnier und Schüler melden sich mit ihrerer Strategie an,
4. Turnier wird ausgeführt und die Spiele analysiert
5. Schüler verbessern ihre Strategien (Gewinnstrategie im Training verfügbar?), erneutes Turnier

\section{Anforderungsanalyse}

Welche Aspekte von CT werden noch nicht vom CT Game Studio unterstützt,

z.B. Evaluation, Kreislauf, selber dem Gegnerroboter Strategie auswählen

Anforderungen aus Szenario und Litaratur extrahieren

\section{Offener Spielmodus}

- Was wollen wir mit einem offenen Spielmodus erreichen?
  - Herausforderungen bieten, die an die Fähigkeiten anknüpfen, die man im Storymodus lernt,
    also CT-Fähigkeiten fordert und fördert, so dass ein intelligenter Einsatz dieser Fähigkeiten zu größerem Erfolg führt
  - Fortgeführter Spielspaß nach Beenden des Storymodus
  (- Einbettung in Unterrichtsszenarios, so dass das Spiel einerseits Lernen im Einzelunterricht und Evaluation in der Gruppe unterstützt)
- Inspiriert von RoboCode soll es ein Roboterkampf geben, so dass eine eigens entwickelte Strategie gegen eine gegnerische Antritt
- Nutzen die Bewegungssemantik des Storymodus, um ein zugängliches und nachvollziehbares (computational model/medium/was ist es?) zu bieten, an dem sich Schüler ihre Fähigkeiten auf iterative Weise ausbilden können (->Papert bilden einer Sprache)
- In der Mikrowelt können Abstraktionen entwickelt werden, die einen erfolgreichen Roboterkampf ausmachen, z.B. Ausfinden machen des Gegners, Zielen und Angreifen, strategisch wirksame Positionen einnehmen, den Gegner verfolgen, Ausweichen, Verhalten verändern aufgrund veränderter Spielsituationen
- Das Entwickeln einer Strategie, bzw. die Umsetzung dieser Abstraktionen erfordert andere CT-Fähigkeiten wie Aufspaltung (z.B. Ausweichen = Treffer feststellen + neue Position einnehmen) in und Algorithmisches Denken
  - Gesamte Strategie bildet ein Algorithmus (Initiale Positionierung, Endlosschleife, usw.) mit 
  - z.B. Verfolgen und Angreifen umsetzen in dem man ein einer Schleife scan + schießen + neu orientieren falls Roboter nicht mehr in Sicht + ...)
- Der Gegner-Roboter ist zum einen ein motivierender Faktor aufgrund des Wettbewerbs ("Ich muss überleben/will besser sein"),
  zum anderen ein Spiegelbild des eigenen Roboters und damit Evaluationssubjekt ("Um mich zu verteidigen muss ich verstehen was er macht" und anhand des Gegners sehen wie man selbst vorgehen kann)
- Community-Aspekt durch Turniermodus: Evaluation gegen andere Mitschüler, zusätzliche Motivation -> "besser als meine Mitschüler sein", sich unbekannten Herausforderungen stellen

\subsection{Gameplay}

- Ein zweiter Roboter + zweite Lebensanzeige muss hinzu gefügt werden
- Ziel: Bessere/überlegtere Strategien sollen sich auch im Matchergebnis niederschlagen
- Verschiedene Ansätze sollen zum Erfolg führen können, z.B. effizientes Ausweichen/Positionieren, oder besonders raffiniertes Verfolgen
- Stand zuvor: Sehr schnelle Bewegungen und Scanvorgang ermöglichen dies nicht und erschweren Evaluation (was passiert)
- Verlangsamung der Bewegungen, Scanvorgang muss sich ausbreiten

\subsection{Trainingsmodus}


\subsection{Turniermodus}
