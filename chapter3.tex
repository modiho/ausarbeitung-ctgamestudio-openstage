\chapter{Ansatz}

Im Storymodus des ctGameStudios lernen Schüler grundlegende Konzepte der Programmierung wie
Schleifen, Verzweigungen, Ereignisse, Prozeduren und Funktionen kennen. Der Storymodus ist in sich
geschlossen, da er einen festen Anfang und Ende hat, und im vornherein festgelegte, spezifische
Herausforderungen stellt.

Wir wollen das ctGameStudio erweitern, um Spielern die Möglichkeit und die Motivation dazu zu geben,
die im Storymodus gelernten Fähigkeiten anzuwenden und zu trainieren. Im Gegensatz zum Storymodus
soll dieser Spielmodus offen sein, so dass der Spieler seine Herausforderungen und Lösungsansätze
selbst bestimmen kann. Damit wiederholtes Spielen motiviert wird und die eigene Kreativiät gefördert
wird, sollen Herausforderungen durch sich selbst oder andere generiert werden können. 

Ein offener Spielmodus stellt die Fähigkeit in den Vordergrund, Problemstellungen zu analysieren,
und adäquate Problemlösestrategien zu entwickeln. Dadurch werden speziell die CT-Fähigkeiten der
Abstraktion und der Evaluation gefördert.

Inspiriert vom RoboCode und angelehnt an das letzte Level des Storymodus soll dieser Spielmodus aus
einem Roboterkampf bestehen. Es gilt, eine Strategie zu entwickeln, um die Strategie des Gegners zu
überwinden.

\section{Ein Lernszenario}

Das Spiel soll solche Schüler unterstützen, die die Grundlagen der Programmierung kennen lernen und
ausbauen sollen. Der Kernlehrplan Informatik für Gymnasien und Gesamtschule in der Sekundarstufe II
(\cite{SchulministeriumNRW2014}) legt dar, welche Kompetenzen ausgebildet und Inhalte
thematisiert werden sollen. Im Fokus soll das Inhaltsfeld der Algorithmen stehen, welches im
Lehrplan als "genaue Beschreibung zur Lösung eines Problems" (S. 17) definiert ist. Dies entspricht
der Entwicklung von Problemlösestrategien in der Roboter-Mikrowelt des ctGameStudio.

Folgendes Lernszenario soll darstellen, wie das Spiel die Ziele des Lehrplans unterstützt und in den
Unterricht eingebunden werden kann.

Um die Grundlagen des Spiels sowie die zur Entwicklung von Problemlösestrategien nötigen
Programmierkonzepte kennen zu lernen, spielen Schüler zunächst den Storymodus des ctGameStudio.
Durch Bearbeiten der Level des Storymodus werden die inhaltlichen Schwerpunkte der Analyse, Entwurf
und Implementierung einfacher Algorithmen (S. 23) unterstützt, und anhand dessen die Kompetenzen des
Argumentierens, der Modellierung, der Implementation, des Darstellen und Interpretieren und
Kommunizieren und Kooperieren gefördert.

Mit dem offenen Spielmodus werden die erlenten Kompetenzen vertieft. In Kämpfen zwischen dem eigenen
gegen einen Gegnerroboter entwickeln die Schüler eigene Kampfstrategien. Dabei können sie die
Gegnerstrategie aus eigenen oder vorgefertigten Strategien festlegen, um eine generell anwendbare,
gegen viele Herausforderungen effektive Strategie zu entwickeln. Nachdem die Schüler eine oder
mehrere Strategien entwickelt haben, kann der Lehrer ein Turnier veranstalten, in dem die Strategien
gegeneinander ausgespielt werden. Das Turnier wird an einem gemeinesamen Bildschirm oder Projektion
verfolgt, und liefert eine Diskussionsgrundlage um Strategien gemeinsam zu analysieren und
evaluieren. (Auch hier Kompetenz des Kommunizieren, blabla). In weiteren Durchgängen können Schüler
ihre Strategie verbessern, und weitere Turniere veranstaltet werden.


\section{Offener Spielmodus: RoboStrategist}

Kern des offenen Spielmodus ist der Kampf zwischen zwei Robotern, die vorprogrammierte Strategien
ausführen, mit dem Ziel, den Gegner zu bezwingen. Im Trainingsmodus kann der Spieler eigene
Strategien entwickeln, während der Turniermodus eine Plattform dafür bietet, verschiedene Strategien
gegeneinander zu evaluieren.

Die Roboter eine begrenzte Zahl von Lebenspunkten. Durch Schüsse auf den Gegner können diese
Lebenspunkte reduziert werden. Die Fähigkeiten des Roboters basieren auf der Bewegungssemantik, die
für den Storymodus des ctGameStudio entwickelt wurden.

\todo{bewegen in ctGameStudio Abschnitt in den Grundlagen?}

\begin{itemize}
\item Positionierung auf dem Spielfeld durch die Bewegung um eine Distanz oder bis zu einem Punkt
auf der x- und y-Achse, der Bewegung bis die Wand oder der Gegner berührt wurde, dem Drehen um einen
Winkel, und dem Drehen, um einen Punkt anzupeilen.
\item Angriff durch Abgeben eines Schusses in Blickrichtung.
\item Das Ausfinden machen des Gegners durch einen Scanvorgang in eine Richtung um den nächsten Gegner
  zu finden, der sich in dieser Richtung befindet. Wurde ein Gegner gefunden, kann die Distanz und
  Winkel zum ihm und seine Position gelesen werden.
\item Feststellen von Ereignissen, wie der Fall ob man getroffen wurde, oder ob man mit der Wand oder dem
Gegner kollidiert ist.
\item Lesen der eigenen Attribute, darunter die Position auf dem Spielfeld und die Anzahl an
verbleibenden Lebenspunkten.
\end{itemize}

Anhand dieser Aktionen und der Blockly-Programmiersprache werden die Kampfstrategien gebaut. Im
Verlauf der Entwicklung einer Strategie entwickelt der Spieler kampfbezogene Abstraktionen (?) wie die
strategische Positionierung und Bewegung auf dem Spielfeld um sich nicht finden zu lassen, dem
Ausweichen, nach dem man getroffen wurde, dem Ausfindig machen des Gegners, das Verfolgen des
Gegners nachdem man ihn gefunden hat, und der Angriff. ~~Je besser er diese löst, desto
wahrscheinlicher ist, dass die Strategie erfolgreich ist~~.


\subsection{Trainingsmodus}

-> Der Trainingsmodus fördert eigenständiges, selbst-geleitetes Vorgehen beim Bauen von
Kampfstrategien.- Die selbst-geleitete Problemstellung und Prozess fördert die Fähigkeit der Evaluation
- Zum Einen werden durch komplexe Herausforderungen in Form von unterschiedlichen Gegnerstrategien wird eigenständige
Problemlösung gefördert, und die Programmierfähigkeit damit stärker vertieft.
- Zum Anderen soll Kreativität und Exploration des Spieler gefordert werden, in dem er selbst
auswählt, gegen welche Strategien er kämpft. Dabei soll aus vorgefertigten und selbst gebauten
Strategien ausgewählt können.
- Angelehnt an die letzten beiden Level des Storymodus werden alle CT-Fähigkeiten benötigt und
ausgebildet. Gesamte Strategie bildet ein Algorithmus (Initiale Positionierung, Endlosschleife, usw.) mit 
  - z.B. Verfolgen und Angreifen umsetzen in dem man ein einer Schleife scan + schießen + neu orientieren falls Roboter nicht mehr in Sicht + ...)
- Prozess der Entwicklung bei erstem Öffnen des Trainingsmodus
  1. Die initial geladenene Gegnerstrategie wird ausgeführt (Bonus: Man hat direkt einen Startpunkt
  und weiß wo man anfangen kann...)
  2. Man analysiert die Gegnerstrategie und bildet erste Lösungsansätze (Der Gegner ist .
  3. Man formuliert und implementiert erste Lösungsansätze. Man kann auf das Wissen aus den letzten
  Leveln des Storymodus zurückgreifen, um einen ersten Anhaltspunkt dafür zu haben.
  4. Man führt den Kampf aus und analysiert seinen Verlauf
    - Macht der Roboter das, was ich mit dem Programm erzielen wollte?
    - Führt mein Problemansatz zu Erfolg?
  5. Aufgrund der Analyse verbessert der Spieler seine Strategie.
  6. Schritt 4 und 5 werden wiederholt, bis das Erfolgskriterium erreicht ist.
  7. Der Spieler wählt eine neue Gegnerstrategie, um seine Strategie an einer neuen Strategie zu
  testen.
  8. Schritt 4 bis 7 werden wiederholt, bis der Spieler alle Gegnerstrategien bezwungen hat und mit
  seiner Strategie zufrieden ist.
  9. Der Spieler kann außerdem versuchen, seine eigene Strategie zu besiegen, in dem er sie als
  Gegnerstrategie festlegt.
- Um verschiedene Strategieansätze ausprobieren zu können, kann der Spieler seine Strategien
speichern, kopieren, neue anlegen, und zwischen diesen wechseln.
- Die Umgebung ermöglicht iteratives Vorgehen (Papert Sprache?).


- Was wollen wir mit einem offenen Spielmodus erreichen?
- Das Entwickeln einer Strategie, bzw. die Umsetzung dieser Abstraktionen erfordert andere CT-Fähigkeiten wie Aufspaltung (z.B. Ausweichen = Treffer feststellen + neue Position einnehmen) in und Algorithmisches Denken
  - 
- Der Gegner-Roboter ist zum einen ein motivierender Faktor aufgrund des Wettbewerbs ("Ich muss überleben/will besser sein"),
  zum anderen ein Spiegelbild des eigenen Roboters und damit Evaluationssubjekt ("Um mich zu verteidigen muss ich verstehen was er macht" und anhand des Gegners sehen wie man selbst vorgehen kann)
- Community-Aspekt durch Turniermodus: Evaluation gegen andere Mitschüler, zusätzliche Motivation -> "besser als meine Mitschüler sein", sich unbekannten Herausforderungen stellen

\subsection{Gameplay}

- Ziel: Bessere/überlegtere Strategien sollen sich auch im Matchergebnis niederschlagen, ein intelligenter Einsatz der Programmierfähigkeiten führt zu größerem Erfolg führt
- Verschiedene Ansätze sollen zum Erfolg führen können, z.B. effizientes Ausweichen/Positionieren, oder besonders raffiniertes Verfolgen
- Eval


\subsection{Turniermodus}
      
\section{Anforderungsanalyse}

- Trainingsmodus
  - Gegnerstrategien auswählen (vorgegebene und eigene)
  - Strategiemanagement: Speichern, Laden, Bearbeiten
- Turniermodus
  - Verschiedene Modi (warum?)
  - Verschiedene Optionen
  - Teilnahme
Welche Aspekte von CT werden noch nicht vom CT Game Studio unterstützt,

z.B. Evaluation, Kreislauf, selber dem Gegnerroboter Strategie auswählen

Anforderungen aus Szenario und Litaratur extrahieren
