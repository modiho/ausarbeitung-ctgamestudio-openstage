\chapter{Einleitung}

"Wie Facebook 4 Mio. Datensätze verloren hat" (Fakenews 2017). "AI wird uns zerstören" (Fakemag,
2015). "Big Technology invests 500 M to bring CS into schools" (Mustermag, 2017). Die Informatik
umgibt nahezu alle Menschen, sie transformiert unseren Alltag und unseren Arbeitsplatz. Dies
erfordert eine Gesellschaft, die in Kontrolle der Technologie ist, und weiß, wie damit umzugehen
ist. Dabei fangen wir gerade erst an, die Informatik auch in dieser Reichweite und angemessenen
Umfang in die Schulen zu bringen.

Mit "Computational Thinking" hat Jeanette Wing einen einflussreichen Forschungsansatz gestellt zu
der Frage "Welche Fähigkeiten werden in der Informatik gebraucht?". Unter diesem Begriff haben sich
eine Reihe von Curricula und Werkzeugen gesammelt, die versuchen, diese Fähigkeiten Schülern und
anderen Interessierten näher zu bringen.

Mit RoboPlanet wurde ein Spiel entwickelt, bei dem der Lernende mittels einem zugänglichen visuellen
Programmiersprache einen Roboter programmiert, um verschiedene Spielziele zu erreichen. Das Spiel
besitzt einen Storymodus, bei dem dem Spieler schrittweise neue Programmierkonzepte beigebracht
werden. Zu Ende der Story kann der Roboter komplexe Probleme lösen.

Im Rahmen dieser Arbeit wird RoboPlanet um einen offenen Spielmodus erweitert, bei dem der Spieler
seinen Roboter gegen andere Roboter antreten lässt. Im Training entwickelt der Schüler Strategien,
um die Gegner, die wiederrum verschiedene Kampfstrategien besitzen, zu besiegen. In einem
klassenübergreifenden Wettbewerb werden die entwickelten Roboter gegeneinander antreten. Durch die
Erweiterung soll motiviert werden, und fortgeschrittene, nahezu unbegrenzte Programmierung gefördert
werden.

Im Folgenden wird zunächst dargestellt, wie Computational Thinking durch RoboPlanet gefördert wird.
Dazu wird die Theorie und verwandte Arbeiten vorgestellt. Wir untersuchen die Bestandteile von
Computational Thinking, und die Mechanismen mit denen Schüler diese Bestandteile lernen, bzw. Lernen
dieser Bestandteile gefördert werden. Daraufhin wird exploriert, welchen Mehrwert RoboArena bringt,
welche Konzeption daraus entsteht, und wie dieses technisch umgesetzt wird.

Test \cite{Ikeda1997}

\section{Motivation}

\section{Aufgabenstellung}

\section{Aufbau der Arbeit}
