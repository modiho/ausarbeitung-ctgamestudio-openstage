\chapter{Einleitung}

\section{Motivation}

Computational Thinking bezeichnet Denkprozesse, die das Formulieren von Problemen sowie deren
Lösungen in einer Maschinen-berechenbaren Form vorsehen. Dies geschieht beispielsweise durch
Abstraktion und Repräsentation von Problemen, der Programmierung einer Lösungsstrategie, sowie der
Evaluation der logischer Artefakte (computational artifacts). Das Praxisprojekt ctGameStudio
bietet ein Framework, bei dem Lernende deren Computational Thinking Skills erweitern können. Dabei
lernen sie verschiedene Möglichkeiten der Abstraktion, indem sie einen virtuellen Roboter mit einer
visuellen blockbasierten Programmierung steuern. Dabei wurden verschiedene Probleme in Leveln so
vorstrukturiert, dass Lernende schrittweise an diese Abstraktionen herangeführt werden. Eine
kompetitive open stage (\enquote{Roboterkampf}) kann Lernenden jedoch die Möglichkeit bieten, ihre
erworbenen Kompetenzen anzuwenden, um (unterschiedlich) komplexe Strategien zu entwickeln und zu
evaluieren. Dadurch wird ein stärkerer Fokus auf die Reflexion und Evaluation der eigenen
Programmierartefakte gesetzt und ein größerer Rahmen für Kreativität und Problemorientierung
ermöglicht.

\section{Aufgabenstellung}

Das Ziel dieser Arbeit besteht darin, das CT Game Studio um eine \enquote{open stage} zu erweitern,
bei der ein Roboterkampf (1 gegen 1) möglich sein soll. Dieser Modus besteht aus einem
Trainingsmodus und einem Turniermodus. Im Training können Lernende eigene Strategien
entwickeln und Verwalten und als Turnierstrategie für ihren eigenen Roboter festlegen. Im
Trainingsmodus steht ein Nicht-Spieler-Gegner bereit, dessen Strategie der Lernende selbst festlegen
kann – entweder eine eigene oder eine von drei voreingestellten Strategien. Für den Turniermodus
soll ein Webinterface zur Konfiguration, Teilnahme und Durchführung von Turnieren erstellt werden.

\section{Aufbau der Arbeit}

Zunächst werden theoretische Grundlagen des ctGameStudio und des Open Stage-Modus aufgearbeitet. Im
Ansatz wird erläutert, wie der Open Stage-Modus konzipiert wurde. Im vierten Kapitel wird der neue
Modus anhand von Screenshots vorgestellt. Es folgt eine Darstellung der Änderungen an der
Architektur des ctGameStudios, die zur Implementation des Open Stage-Modus vorgenommen wurden. Es
wird mit einem Fazit und anschließendem Ausblick abgeschlossen.
