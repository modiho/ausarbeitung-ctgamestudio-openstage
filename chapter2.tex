\chapter{Grundlagen}

\section{Computational Thinking}

Wing, 2006, 2008; Selby, 2011; Grover, 2011?

Unter Computational Thinking wird Kognitionsprozess oder Gedankenprozess verstanden, der durch die
Fähigkeit, in Form von Dekomposition, abstrahierend, evaluierend, algorithmisch und generalisierend
zu denken, reflektiert wird. Ziel des Prozesses ist es, ein Problem so dar zu stellen, dass es von
einem Computer gelöst werden kann. Im Folgenden sollen diese Denkweisen im Einzelnen vorgestellt
werden.

CT-Aktivitäten (z.B. Debugging)

- Algorithmisches Denken

...
(Scratch als Kontextualisierung CT in Education)

CT hat seinen Weg in den Schulalltag gefunden.




Computational Thinking (CT)

Aktivitäten/Fähigkeiten
Computational Artefacts





\section{Turtle Geometrie}

- Vereinfachte Programmiersprache, entwickeln eines eigenes Vokabulars ermöglicht
- Mikrowelten
- Body-syntonic
- Iteratives Vorgehen
- Bewegungssemantik


\section{Game-based learning von Computational Thinking}

- Spiele häufig genutzt als Lernumgebungen (Kara, Program you robot, etc.)
- Rieber, 1996 -> Spiele können aufgrund verschiedener Aspekte (fantasy, control, competition, etc.) Mikrowelten und Simulationen in einer Weise bereit stellen, in der Lerner internal motiviert sind Spielziele und damit Lernerfolge zu erzielen 
- low floor, high-ceiling

\subsection{Program Your Robot}
 
\cite{Kazimoglu2012}

Program your robot als Spiel das Computational Thinking skills direkt auf Spielelemente mappt.

\subsection{RoboCode}

- Paper referenzieren (problem based learning)

\cite{RoboCodeWebsite}

- Herkunft, RoboCode als Beispiel für ein Programmierspiel und offenen Spielmodus
- Beschreibung des Gameplays
- Community: Wiki, Austausch von Strategien, Turniere.


\subsection{CT Game Studio}

\cite{Werneburg2018}

- Guidance
- Microworld
- block-based programming
- Storymodus, Level mit Fokus auf spezifischen oder Kombination von CT Skills
- Evaluation unterstützt durch direktes Ausführen des Codes/Abbildung des Verhaltens/wiederholtes Ausführen, Highlight von ausgeführten Codeabschnitten, Erreichen von Zielen
- Kreislauf? (Lightbot?)
- analytics component
- Ausblick open stage -> Strategien
- Man könnte auch hier scratch nennen
- Phaser game framework, Keystone server web framework and database, blockly-Bibliothek

Generell soviel wie möglich von vorher aufgreifen um Connections zu ziehen

