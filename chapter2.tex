\chapter{Grundlagen}

\section{Computational Thinking}

Unter Computational Thinking wird Kognitionsprozess oder Gedankenprozess verstanden, der durch die
Fähigkeit, in Form von Dekompositin, abstrahierend, evaluierend, algorithmisch und generalisierend
zu denken, reflektiert wird. Ziel des Prozesses ist es, ein Problem so dar zu stellen, dass es von
einem Computer gelöst werden kann. Im Folgenden sollen diese Begriffe

\section{CT Game Studio}

CT Game Studio ist ein Ansatz zur Vermittlung von Computational Thinking Kompetenzen.

Computational Thinking

Game-based-learning (Konstruktivismus)

Blockbasierte Programmierung


\section{Verwandte Projekte}

Das CT Game Studio ist einer von vielen Ansätzen, Computational Thinking in Spielumgebungen zu
lehren. Im Folgenden soll eine Auswahl von solchen Spielen vorgestellt werden.


