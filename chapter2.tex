\chapter{Grundlagen}

\section{Computational Thinking}

Unter Computational Thinking wird Kognitionsprozess oder Gedankenprozess verstanden, der durch die
Fähigkeit, in Form von Dekomposition, abstrahierend, evaluierend, algorithmisch und generalisierend
zu denken, reflektiert wird. Ziel des Prozesses ist es, ein Problem so dar zu stellen, dass es von
einem Computer gelöst werden kann. Im Folgenden sollen diese Denkweisen im Einzelnen vorgestellt
werden.

CT-Aktivitäten (z.B. Debugging)
...



\section{CT Game Studio}

CT Game Studio ist eine Lernumgebung zur Vermittlung von Computational Thinking Kompetenzen.

Game-based-learning

Blockbasierte Programmierung

...

\section{Effektives Game-based-learning}

Welche Eigenschafen weisen Spiele auf, die sich positiv auf Lernerfolg auswirken?

\subsection{Motivation und Engagement}

\textit{Games engage players on three main fronts: 1) The structure of the game provides motivation and the
urge to solve problems for the problem's sake alone. 2) The backstory or narrative provides the
believability or authenticity of engagement. 3) Characterization makes the player's role in the
narrative believable so that the player can engage fully in the game}

Flow besteht aus Herausforderung und Fähigkeiten und wirkt sich positiv auf das Engagement aus

Engagment wirkt sich positiv auf Lernerfolg aus

Wiederholtes Spielen wirkt sich positiv auf Lernerfolg aus -> Wiederspielbarkeit ist wichtig

\subsection{Guided vs Discovery Learning}

...

\subsection{Social Environments und Gruppenstruktur}

- Sharing von Ergebnissen
- Kollaboration/Kompetitiv/vs single player (Goal structures?)

\subsection{Scaffolding/Integration in Klassenraum}

Use-create-modify

Debriefing?


\section{Verwandte Projekte}

Das CT Game Studio ist einer von vielen Ansätzen, Computational Thinking in Spielumgebungen zu
lehren. Im Folgenden soll eine Auswahl von solchen Spielen vorgestellt werden.

\subsection{Blockly Games}

Blockly Games ist eine webbasierte Kollektion von sechs Minispielen, mit unterschiedlichen
Mechaniken.Die Programmierung ist blockbasiert, wobei die verfügbaren Blöcke an die Mechanik des
jeweiligen Spiels angepasst sind.

In \emph{Puzzle} wird durch ein einfaches Assoziationsspiel die Grundlagen der Blockbasierten
Programmierung näher gebracht.

In den weiteren Spielen wird eine Spielfigur gesteuert. (Erklärung der
UI/Ausführen-Reset-Zyklus/Instruktion). Dabei bestehen die folgenden vier Spiele aus jeweils zehn
Leveln, die den Spieler schrittweise an neue Konzepte heran führen und größere, komplexere
Herausforderungen stellt.

In \emph{Maze}...

\begin{itemize}
\item Maze hat Limit an Blocks um Abstraktion zu fordern, und ein Auswahlmenü zur Auswahl eines Avatars
(Personalisierung)
\item Levels können frei ausgewählt werden, das Spiel zeigt an welche Levels abgeschlossen wurden
\item Der resultierende Javascript Code wird nach erfolgreichem Abschluss des Levels angezeigt
\item Start/Reset
\item Neue Konzepte und Interaktionsmöglichkeiten werden durch Hinweise eingeführt und erklärt. Teilweise
tauchen auch Hilfestellungen auf, wenn das angenommen wird, dass der Spieler nicht weiter kommt.
Bird, Turtle, Movie und Pond Tutor starten jedes Level mit einer Instruktion zur Lösung des Levels.
Sie kann über einen dedizierten Hilfe-button erreicht werden.
\end{itemize}

\subsection{RoboCode und RoboBuilder}

...

\subsection{Dragon Architect}

...

Der Vollständigkeit ein Spiel vorstellen, dass nicht Blockbasiert ist? Z.B. Lightbot oder Program
Your Robot

