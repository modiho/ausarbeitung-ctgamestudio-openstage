%appendixprefix sorgt dafür, dass "Anhang" vor jedem Anhang steht
%openany sorgt dafür, dass ein Kapitel auf jeder Seite beginnen kann
%        nicht nur rechts
%bibtotoc sorgt dafür, dass das Literaturverzeichnis automatisch im
%         Inhaltsverzeichnis erscheint
\documentclass[fontsize=11pt,paper=a4,parskip=half,draft=false,appendixprefix,bibliography=totoc,
headings=normal]{scrbook}


%
% neues if definieren, um zwischen PDF und DVI entscheiden zu können.
%
%\newif\ifpdf
%\ifx\pdfoutput\undefined
%\pdffalse %not PDFLaTeX
%\else
%\pdfoutput=1
%\pdftrue
%\fi
%\tracingstats=2
%\usepackage{layout}
% german language support (hyphenation etc)  
%\usepackage{ngerman}
\usepackage[ngerman]{babel}
% support for latin1 characters. That means you can enter ä ö ü ß directly
% no need for "a "u "ü "s anymore
%\usepackage[latin1]{inputenc}
\usepackage[utf8]{inputenc}
% provides the \url{} command to pretty print urls
\usepackage{url}

% needed for a german bibliography-style (s. below)
\usepackage[german]{babelbib}

% allows text flowing around figures.
\usepackage{wrapfig}

% allows to \includegraphics
\usepackage{graphicx}

% defines some standard colornames like "black" etc.
\usepackage{color}

% allows to color tablecells
\usepackage{colortbl}

% provides an easier interface to if-then-else constructs in 
% custom macros
\usepackage{ifthen}

% allows tables to break over pages.
\usepackage{supertabular}

% allows to have different kinds paper orientations in the same pdf-documnent
\usepackage{pdflscape}

% allows to specify absolute texpos for textboxes. This is generally only important for the titlepage
\usepackage[absolute]{textpos}

% allows to enumerate different figures with a) b) in the same figure-environment.
\usepackage{subfigure}

% finetune the gaps between figure and text in the subfigure environment (basically close the gap as much as possible)
\renewcommand{\subfigtopskip}{0pt}
\renewcommand{\subfigbottomskip}{0pt}

% some color definitions for the pdf-statements below
\definecolor{mygrey}{rgb}{0.45,0.45,0.45}
\definecolor{mydarkgrey}{rgb}{0.2,0.2,0.2}
\definecolor{red}{rgb}{1.0,0.33,0.33}
\definecolor{orange}{rgb}{1.00,0.73,0.33}
\definecolor{yellow}{rgb}{0.95,0.92,0.}
\definecolor{lightgreen}{rgb}{0.3,0.95,0.46}
\definecolor{titleblue}{rgb}{0.03,0.10,0.46}

%\ifpdf
% Metadata and configuration of the pdf output:
% Do not forget to enter the correct title, author, subject und keywords

% For screen viewing it is nice to have references marked in a slightly different
% color than the rest of the text. Since they will be hyperlinks to the 
% referenced objects.
\usepackage[pdftex,
             pdftitle={Erweiterung des CT Game Studios um einen "Open Stage" Modus},
             colorlinks,
%             linkcolor={mydarkgrey},
%             citecolor={mygrey},
%             urlcolor={black}
             linkcolor={lightgreen},
             citecolor={mygrey},
             urlcolor={blue},
             plainpages={false},
             bookmarksnumbered={true},
             pdfauthor={Daniel Rose},
             pdfsubject={},
             pdfkeywords={},
             pdfstartview={FitBH}]{hyperref}

% For the final printouts (remember - you need at least three - one for each examiner and one for the archive 
% [ This might have changed - so contact the "Prüfungsamt" about the current regulations !! ] - it is better
% to have all text in the same color (namely black).
% 
%\usepackage[pdftex,
%            pdftitle={},
%            colorlinks,
%            linkcolor={black},
%            citecolor={black},
%            urlcolor={black},
%            plainpages={false},
%            bookmarksnumbered={true},
%            pdfauthor={},
%            pdfsubject={},
%            pdfkeywords={},
%            pdfstartview={FitBH}]{hyperref}
\pdfcompresslevel=9
%\fi

% some configuration for the amount of text on a single page
\usepackage{typearea}\areaset[1.5cm]{418pt}{658pt}
\setlength{\headheight}{37pt}

% To avoid nasty mistakes like having comments directly in the textflow
% the following \todo macro was defined. With that you can enter
% \todo{What I still have to do here} 
% inside of your text and a marker will appear at the page's margin with the 
% text "What I still have to do here".
% The first line activates this feature. If you comment it out and uncomment
% the second line below there will be no error messages and no todos will be show
% anymore. So - even if you have forgotten to delete one of them - they will not appear
% in the final printout. 
\newcommand{\todo}[1]{\marginpar{\textcolor{red}{ToDo:} #1}}
%\newcommand{\todo}[1]{}

% We recommend to split your document into several files. Usually one for every chapter is a 
% good idea. If you follow this guideline (how to assemble these files in a single document
% see two paragraphs below) you will be able to single out chapters via the \includeonly{}
% command. Using this mechanism page numbering and references of the full run before will be
% preserved. This also nice, if your latex run tends to get slow and you need to fine tune 
% some formatting in one chapter - just include that one. The rest (or at least the ones before
% the one currently under construction) will remain untouched. This means a boost in compilation time.
%\includeonly{chapter2}



%\usepackage[style=authoryear-ibid]{biblatex}
\usepackage[german=quotes]{csquotes}

%\bibliography{references}

\begin{document}
% the next two lines influence the detailedness of the table of contents
% and to what structure depth numbers are written before sections/subsections/paragraphs
% You should not touch this
\setcounter{tocdepth}{3}
\setcounter{secnumdepth}{3}
\frontmatter
% here the titlepage is included. Look into the file "titelseite.tex" to 
% adapt it to your needs (name, title etc.)
% Titelseite braucht folgenden  Eintrag
% \usepackage[absolute]{textpos}
% textpos ist nicht Bestandteil von tetex
% kann aber von dante heruntergeladen werden
\begin{titlepage}
\vspace*{-1cm}
\newlength{\links}
\setlength{\links}{0.9cm}
\setlength{\TPHorizModule}{1cm}
\setlength{\TPVertModule}{1cm}
\textblockorigin{0pt}{0pt}

\sffamily
\LARGE

\begin{textblock}{16.5}(2.8,2.6)
 \hspace*{-0.25cm} \textbf{UNIVERSITÄT DUISBURG-ESSEN} \\
 \hspace*{-1.15cm} \rule{5mm}{5mm} \hspace*{0.05cm} FAKULTÄT FÜR INGENIEURWISSENSCHAFTEN\\
 \large{}ABTEILUNG INFORMATIK UND ANGEWANDTE KOGNITIONSWISSENSCHAFT\\
\end{textblock}


%Hier Titel, Name, und Matrikelnummer eintragen, \\ make a newline
\begin{textblock}{14.5}(3.2,8.5)
  \large
{ \textbf{Bachelorarbeit}} \\[1cm]
{\LARGE \Large\textbf{Erweiterung des CT Game Studios um einen \enquote{Open Stage} Modus}} \\[1.3cm]
Daniel Rose\\
Matrikelnummer: 2270435
\end{textblock}



\begin{textblock}{10}(10.5,17.5)
\includegraphics[scale=1.0]{unilogo.pdf}\\
\normalsize
\raggedleft
Abteilung Informatik und angewandte Kognitionswissenschaft \\
Fakultät für Ingenieurwissenschaften \\
Universität Duisburg-Essen \\[2ex]

\today\\[15ex]
\raggedright
% Supervisors
{\textbf Betreuer:} \\
Prof. Dr. H.~U.~Hoppe\\
Sven Manske\\
Sören Werneburg\\
\end{textblock}

\end{titlepage}

\tableofcontents
\listoffigures
\mainmatter

% To assemble the whole document
% Please be aware that each file will begin on a new page
% therefore chapters should be put into such a file.
% There cannot be an include statement inside of an "included" file.
% So if you want to further divide your document - use \input inside of 
% the included files. \input will not begin on a new page.
\chapter{Einleitung}

TODO

\section{Motivation}

\section{Aufgabenstellung}

\section{Aufbau der Arbeit}


\chapter{Grundlagen}

\section{Computational Thinking}

Wing, 2006, 2008; Selby, 2011; Grover, 2011?

Unter Computational Thinking wird Kognitionsprozess oder Gedankenprozess verstanden, der durch die
Fähigkeit, in Form von Dekomposition, abstrahierend, evaluierend, algorithmisch und generalisierend
zu denken, reflektiert wird. Ziel des Prozesses ist es, ein Problem so dar zu stellen, dass es von
einem Computer gelöst werden kann. Im Folgenden sollen diese Denkweisen im Einzelnen vorgestellt
werden.

CT-Aktivitäten (z.B. Debugging)

- Algorithmisches Denken

...
(Scratch als Kontextualisierung CT in Education)

CT hat seinen Weg in den Schulalltag gefunden.




Computational Thinking (CT)

Aktivitäten/Fähigkeiten
Computational Artefacts





\section{Turtle Geometrie}

- Vereinfachte Programmiersprache, entwickeln eines eigenes Vokabulars ermöglicht
- Mikrowelten
- Body-syntonic
- Iteratives Vorgehen
- Bewegungssemantik


\section{Game-based learning von Computational Thinking}

- Spiele häufig genutzt als Lernumgebungen (Kara, Program you robot, etc.)
- Rieber, 1996 -> Spiele können aufgrund verschiedener Aspekte (fantasy, control, competition, etc.) Mikrowelten und Simulationen in einer Weise bereit stellen, in der Lerner internal motiviert sind Spielziele und damit Lernerfolge zu erzielen 
- low floor, high-ceiling

\subsection{Program Your Robot}
 
\cite{Kazimoglu2012}

Program your robot als Spiel das Computational Thinking skills direkt auf Spielelemente mappt.

\subsection{RoboCode}

- Paper referenzieren (problem based learning)

\cite{RoboCodeWebsite}

- Herkunft, RoboCode als Beispiel für ein Programmierspiel und offenen Spielmodus
- Beschreibung des Gameplays
- Community: Wiki, Austausch von Strategien, Turniere.


\subsection{CT Game Studio}

\cite{Werneburg2018}

- Guidance
- Microworld
- block-based programming
- Storymodus, Level mit Fokus auf spezifischen oder Kombination von CT Skills
- Evaluation unterstützt durch direktes Ausführen des Codes/Abbildung des Verhaltens/wiederholtes Ausführen, Highlight von ausgeführten Codeabschnitten, Erreichen von Zielen
- Kreislauf? (Lightbot?)
- analytics component
- Ausblick open stage -> Strategien
- Man könnte auch hier scratch nennen
- Phaser game framework, Keystone server web framework and database, blockly-Bibliothek

Generell soviel wie möglich von vorher aufgreifen um Connections zu ziehen


\chapter{Ansatz}

Im Storymodus des ctGameStudios lernen Schüler grundlegende Konzepte der Programmierung wie
Schleifen, Verzweigungen, Ereignisse, Prozeduren und Funktionen kennen. Der Storymodus ist in sich
geschlossen, da er einen festen Anfang und Ende hat, und im vornherein festgelegte, spezifische
Herausforderungen stellt.

Wir wollen das ctGameStudio erweitern, um Spielern die Möglichkeit und die Motivation dazu zu geben,
die im Storymodus gelernten Fähigkeiten anzuwenden und zu trainieren. Im Gegensatz zum Storymodus
soll dieser Spielmodus offen sein, so dass der Spieler seine Herausforderungen und Lösungsansätze
selbst bestimmen kann. Damit wiederholtes Spielen motiviert wird und die eigene Kreativiät gefördert
wird, sollen Herausforderungen durch sich selbst oder andere generiert werden können. 

Ein offener Spielmodus stellt die Fähigkeit in den Vordergrund, Problemstellungen zu analysieren,
und adäquate Problemlösestrategien zu entwickeln. Dadurch werden speziell die CT-Fähigkeiten der
Abstraktion und der Evaluation gefördert.

Inspiriert vom RoboCode und angelehnt an das letzte Level des Storymodus soll dieser Spielmodus aus
einem Roboterkampf bestehen. Es gilt, eine Strategie zu entwickeln, um die Strategie des Gegners zu
überwinden.

\section{Lernszenario}

Das Spiel soll solche Schüler unterstützen, die die Grundlagen der Programmierung kennen lernen und
ausbauen sollen. Der Kernlehrplan Informatik für Gymnasien und Gesamtschule in der Sekundarstufe
II\footnote{Kernlehrplan Informatik für Gymnasien und Gesamtschule in der Sekundarstufe II:
\cite{SchulministeriumNRW2014}} definiert als Inhaltsfeld die Entwicklung von Algorithmen, welche
als "genaue Beschreibung zur Lösung eines Problems" (S. 17) definiert werden. Dies entspricht der
Entwicklung von Problemlösestrategien in der Roboter-Mikrowelt des ctGameStudio, und zeigt, dass das
dieses Spiel die Anwendung in der Sekundarstufe II unterstützen sollte.

Folgendes Lernszenario soll darstellen, wie das Spiel die Ziele des Lehrplans unterstützt und in den
Unterricht eingebunden werden kann.

Um die Grundlagen des Spiels sowie die zur Entwicklung von Problemlösestrategien nötigen
Programmierkonzepte kennen zu lernen, spielen Schüler zunächst den Storymodus des ctGameStudio.
Durch Bearbeiten der Level des Storymodus werden die im Kernlehrplan definierten inhaltlichen
Schwerpunkte der Analyse, Entwurf und Implementierung einfacher Algorithmen (S. 23) unterstützt, und
anhand dessen die Kompetenzen des Argumentierens, der Modellierung, der Implementation, des
Darstellen und Interpretieren und Kommunizieren und Kooperieren gefördert.

Mit dem offenen Spielmodus werden die erlenten Kompetenzen vertieft. In Kämpfen zwischen dem eigenen
gegen einen Gegnerroboter entwickeln die Schüler eigene Kampfstrategien. Dabei können sie die
Gegnerstrategie aus eigenen oder vorgefertigten Strategien festlegen, um eine generell anwendbare,
gegen viele Herausforderungen effektive Strategie zu entwickeln. Nachdem die Schüler eine oder
mehrere Strategien entwickelt haben, kann der Lehrer ein Turnier veranstalten, in dem die Strategien
gegeneinander ausgespielt werden und eine Rangliste entsteht. Das Turnier wird an einem gemeinesamen
Bildschirm oder Projektion verfolgt, und liefert eine Diskussionsgrundlage um Strategien gemeinsam
zu evaluieren. In weiteren Durchgängen können Schüler ihre Strategie verbessern, und weitere
Turniere veranstaltet werden.


\section{Offener Spielmodus: RoboStrategist}

Kern des offenen Spielmodus ist der Kampf zwischen zwei Robotern, die vorprogrammierte Strategien
ausführen. Die Strategien sind Algorithmen zur Koordination der Fähigkeiten des Roboters, um den
Gegner Schaden zuzufügen, und eigenen Schaden zu vermeiden. Eine erfolgreiche Strategie besteht aus
effektiver Positionierung des Roboters auf dem Spielfeld, dem Ausweichen von Schüssen nachdem man
getroffen wurde, dem Ausfinding machen und Verfolgen des Gegners, und das Zielen und Schießen auf
den Gegner.


\subsection{Training}

Der Training fördert eigenständiges, selbst-geleitetes Lösen von Problemen, und bietet damit
eine Plattform, seine Programmierfähigkeiten zu vertiefen. Zum Einen werden komplexe
Herausforderungen in Form von unterschiedlichen Gegnerstrategien gestellt. Der Spieler
wendet Abstraktionen wie Schleifen, Verzweigungen, Variablen, Prozeduren und Funktionen an, um einen
Algorithmus zu entwickeln. Der Algorithmus ist das Code-Artefakt, das gegen das Problem in Form der 
Gegnerstrategie evaluiert wird.

Zum Anderen kann der Spieler eigenständig wählen, wie er bei der Entwicklung der
Problemlösestrategie vorgeht, in dem er selbst wählt, gegen welche Strategien er seine Strategie
testet. Im Rahmen dieser Arbeit sollen dafür dafür drei vorgefertigte Gegnerstrategien verfügbar
sein. Die Strategien unterscheiden sich in ihren Ansätzen, ihrer Komplexität, und der Schwierigkeit,
sie zu besiegen.

Die erste Strategie, \enquote{Verwirrt}, (Abb. \ref{strategie-verwirrt}) ist dazu gestaltet, einen
einfachen Einstieg in die Strategieentwicklung zu geben. Der Roboter bewegt sich in zufälligen
Zeitabständigen auf zufällig gewählte Positionen, und gibt zwischendurch Schüsse in zufällige
Richtungen ab. Um den Roboter zu besiegen, muss der Spieler eine Strategie entwickeln, die den
Gegner immer wieder auffindet, und Schüsse in seine Richtung feuert. Dadurch, dass kein
Zielen auf den Roboter des Spielers besteht, muss die Lösungsstrategie nicht besonders effizient
sein. Auch um die eigene Positionierung muss sich der Spieler keine Gedanken machen.

\begin{figure}
  \centering
  \label{strategie-verwirrt}
  \includegraphics[scale=1, keepaspectratio]{figures/strategy-verwirrt.png}
  \caption{Blockly-Sicht der \enquote{Verwirrt}-Strategie}
\end{figure}

Die zweite Strategie, \enquote{Wandkrabbler} (Abb. \ref{strategie-wandkrabller}), agiert in einem
vorsehbaren Muster. Der Roboter bewegt sich an am Spielfeldrand entlang und scannt in kleinen
Abständen voneinander entgegengesetzt der Wand nach dem Gegner. Wurde dieser entdeckt, schießt er
solange, bis der Gegner seine Position ändert. Diese Strategie erfordert vom Spieler eine Strategie
zu entwickeln, in der unterschiedliche Postionen eingenommen werden, um die Entdeckung durch den
Gegner zu verzögern und im Fall eines Treffers weiteren Schüssen auszweichen. Die erhöhte Gefahr
durch den Wandkrabbler erfordert auch einen effizienten Scanvorgang, so dass man den Gegner öfter
und schneller auffindet und Schaden zufügt, als er es tut.

\begin{figure}
  \centering
  \label{strategie-wandkrabbler}
  \includegraphics[scale=1, keepaspectratio]{figures/strategy-wandkrabbler.png}
  \caption{Blockly-Sicht der \enquote{Wandkrabbler}-Strategie}
\end{figure}

Die dritte Strategie, \enquote{Eckenschütze} (Abb. \ref{strategie-sniper}), ist die
gefährlichste aller Strategien. Zunächst hat sie den effizientesten Scanmechanismus. Dazu bewegt der
Roboter sich zwischen den Ecken des Spielfelds, und scannt von den Ecken aus in kleinen Schritten
das gesamte Spielfeld ab. So ist die Chance groß, innerhalb eines kurzen Zeitraums den Gegner
aufzufinden. Wurde der Gegner entdeckt, werden so lange Schüsse abgegeben, bis der Gegner seine
Position ändert. Die Strategie implementiert zudem einen Ausweichmechanismus, so dass der Roboter in die
nächste Ecke wechselt, wenn er getroffen wurde. Um diese Strategie zu besiegen erfordert ständige
Bewegung auf dem Spielfeld, um den Scanvorgängen zu entgehen. Wahrscheinlich sollte sie ein eigenen
Ausweichmechanimus implementieren, und einen effiziente Scanvorgang beinhalten.

\begin{figure}
  \centering
  \label{strategie-sniper}
  \includegraphics[scale=1, keepaspectratio]{figures/strategy-sniper-1.png}
  \includegraphics[scale=1, keepaspectratio]{figures/strategy-sniper-2.png}
  \caption{Blockly-Sicht der \enquote{Eckenschütze}-Strategie}
\end{figure}

Wie bereits vorgestellt sollte eine Mikrowelt nach Papert ein iteratives, selbst gesteuertes Vorgehen
erlauben, um die inherente Kreativität und Explorationswillen des Menschen zu nutzen. In diesem
Sinne wollen wir dem Spieler die Möglichkeit geben, als Gegnerstrategien neben den vorgefertigten
auch selbst erstellte Strategien zu wählen. Um seine Strategie zu verbessern, könnte der Spieler
durch eine Analyse Schwachpunkte seiner Strategie feststellen, und versuchen diese mit einer neuen
Strategie konkret auszunutzen.

Das eigenständige, selbst-geleitete Lösen von Problemen soll einen kreativen Lösungsprozess des
Spielers fordern, ein exploratives Vorgehen motivieren, und damit verglichem mit dem geleiteten
Prozess des Storymodus einen höheren Lerneffekt erzielen.

Folgendes Schema stellt eine neue Strategieentwicklung dar.
% und zeigt, dass die Entwicklung einer Strategie im Training den drei Stufen des Compuational Thinking Prozesses \ref{repenning_principles_2017}

\begin{enumerate}
\item Die initial geladenene Gegnerstrategie wird ausgeführt.
\item Man analysiert die Gegnerstrategie und bildet erste Lösungsansätze (Problem Formulation).
\item Man formuliert und implementiert erste Lösungsansätze (Solution Expression). Man kann auf das Wissen aus den letzten
Leveln des Storymodus zurückgreifen, um einen ersten Anhaltspunkt dafür zu haben.
\item Man führt den Kampf aus und analysiert seinen Verlauf (Solution Execution \& Evaluation). Die Evaluationskriterien sind, ob der Roboter
das macht, was man mit dem Programm erzielen wollte, und ob der Problemansatz zum Erfolg führt.
\item Aufgrund der Analyse verbessert der Spieler seine Strategie.
\item Schritt 4 und 5 werden wiederholt, bis der Gegner wiederholt geschlagen werden kann.
\item Der Spieler wählt eine neue Gegnerstrategie.
\item Schritt 4 bis 7 werden wiederholt, bis der Spieler alle Gegnerstrategien hat. Der Spieler kann außerdem
versuchen, seine eigene Strategie zu besiegen, in dem er sie als Gegnerstrategie festlegt.
\end{enumerate}

Um verschiedene Strategieansätze ausprobieren zu können, und in mehreren Sitzungen an den Strategien
arbeiten zu können, kann der Spieler seine Strategien speichern, kopieren, neue anlegen, und
zwischen diesen wechseln.


\subsection{Turniere}

Turniere sind spannend und erhöhen Spielspaß und damit die Motivation, das Spiel zu spielen. Der
kompetitive Aspekt wird Spieler dazu motivieren, gute Strategien zu entwickeln. Die
Turnierdurchführung selber bietet einen Vergleich zwischen Strategien und ist so
Diskussionsgrundlage für verschiedene Strategieansätze und Verbesserungen.

Da es verschiedene Möglichkeiten gibt, ein Turnier durchzuführen, und diese ihre eigenen Vor- und
Nachteile haben, wollen wir dem Turnierveranstalter die Wahl eines Turniersystems geben. In dieser
ersten Version des Open Stage-Modus wollen wir die zwei gebräuchlisten Turniersysteme zur Wahl
stellen.

Das Jeder-gegen-Jeden-System ist die ausführlichste Weise, ein Turnier durchzuführen. Ein Beispiel
für diese Turnierart ist die deutsche Fußballbundesliga. Um in diesem System eine Rangfolge zwischen
Teilnehmern zu erzielen, tritt jeder Teilnehmer gegen jeden anderen Teilnehmer an und zählt die
Siege, die erreicht wurden. Vorteil ist, dass diese Rangfolge erschöpfend ist. Nachteil dieses
Modus ist die quadratisch steigende Anzahl von Kämpfen ($n x n$ Kämpfe bei $n$ Teilnehmern),
die durchgeführt werden müssen. Für die Evaluation von Strategien im Open Stage-Modus bietet sich das
Jeder-gegen-Jeden-System an, wenn eine geringe Anzahl von Teilnehmern besteht.

Das KO-System ist eine effizientere Weise, ein Turnier durchzuführen. Ein Beispiel für diese
Turnierart ist die KO-Phase eines Fußball-Weltmeisterschaft. Das Turnier verläuft in Runden, wobei
in der ersten Runde zufällig ausgewählte Pärchen gegeneinander antreten. Während der Verlierer eines
Kampfes vom Turnier ausscheidet, geht der Sieger in die nächste Runde. Vorteil dieses Turnieres ist,
dass gegenüber dem Jeder-gegen-Jeden-System weitaus weniger Kämpfe ausgeführt werden ($n - 1$ Kämpfe
bei $n$ Teilnehmern). Außerdem hat dieser Modus einen interessanteren Spannungsbogen, da bei jedem
Kampf ein Spieler ausscheidet, und nach jeder Runde die Anzahl der Spieler durch zwei geteilt wird,
bis es am Ende ein spannendes Finale gibt. Nachteil dieses Systems ist, dass die aus dem Turnier
resultierende Rangfolge nicht zwangsläufig repräsentativ für die tatsächliche Rangfolge der
Strategien ist. So mag es beispielsweise sein, dass die Gewinnerstrategie bei einem Turnier mit acht
Teilnehmern eigentlich gegen vier der sieben Gegner verlieren würde, durch die zufällige initale
Paarung jedoch auf die drei übrigen Strategien getroffen ist. Während er in dieser Paarung das
Turnier gewinnen könnte, wäre er im Jeder-gegen-Jeden-Modus nicht zwangsläufig an oberster Stelle
der Rangliste. Ein weiterer Nachteil ist, dass ein faires Turnier im KO-Modus mit mehr als zwei
Teilnehmern eine Teilnehmeranzahl haben muss, die durch vier teilbar ist. Da dies nicht immer
möglich ist, wollen wir dem Veranstalter die Möglichkeit geben, fehlende Teilnahmen durch selbst
festgelegte Strategien aufzufüllen. Für die Evaluation von Strategien im Open Stage-Modus bietet
sich bei einer hohen erwarteten Anzahl von Teilnehmern an, und der Unterhaltungswert des Turniers
eine große Rolle spielt. 

Wenn Strategien ineffektiv sind, kann es sein, das in einem Kampf innerhalb eines vertretbaren
Zeitraums kein Sieger festegestellt werden kann. Deshalb soll bei Erstellung des Turniers eine
maximale Rundendauer festgelegt werden können. Im Jeder-gegen-Jeden-Modus geht der Kampf nach Ablauf
der Zeit unentschieden aus. Im KO-Modus muss ein Sieger festgestellt werden. Dauzu gewinnt in
diesem Modus nach Ablauf dieser Zeit der Spieler, dessen Roboter weniger Schaden genommen hat. Sind
hier beide Spieler gleich, kann der Turnierleiter selbst einen Sieger festlegen.

Da bereits zu vor Ablauf der Zeit absehbar sein kann, dass es zu keinem Sieger kommen wird, soll es
außerdem es die Möglichkeit geben, eine Runde abzubrechen. Der Turnierleiter kann den Kampf
wiederholen, oder das Ergebnis selbst bestimmen.

Das Resultat mehrerer Kämpfe mit den selben Strategien kann aufgrund der zufälligen Startposition
der Roboter unterschiedlich sein. Um die tatsächliche bessere Strategie zu ermitteln, könnte es
nötig sein, einen Kampf aus mehreren Runden bestehen zu lassen. Ein Spieler würde dann den Kampf
gewinnen, der zuerst zwei oder drei Runden gewonnen hat. Auch hierfür soll es bei Erstellung des
Turnieres eine Option geben.

Die Turnierfunktion ist so gestaltet, dass die Teilnehmer nicht bei Erstellung des Turniers bekannt
sein müssen. Dazu wird bei Erstellung einen Zugangscode generiert. Spieler, die diesen Code kennen,
können sich mit einer ihrer im Training erstellten Strategien am Turnier anmelden.

In einer Turnierliste kann der Turnierersteller die Namen und Strategienamen der Teilnehmer
einsehen, und das Turnier starten. Um die Durchführung des Turniers zu verfolgen muss der Bildschirm
des Turniererstellers betrachtet werden.

Eine Turnierübersicht zeigt den Verlauf und Ranglisten des Turniers. Die Ansichten unterscheiden
sich dabei je nach Turniersystem. Im Jeder-gegen-Jeden-Modus werden vergangene Kämpfe, noch
anstehende Kämpfe, und eine Tabelle mit der Anzahl von Siegen pro Spieler angezeigt. Im KO-Modus
wird der Turnierbaum angezeigt. Der Turnierersteller kann den jeweils nächsten Kampf über einen
Button starten. Das Spielfeld wird geladen und der Kampf ausgeführt. Nach Ende des Kampfes wird die
aktualisierte Turnierübersicht dargestellt.


\section{Anforderungsanalyse}

Zur Implementation des Open Stage-Modus muss das Benutzerinterface um Dialoge und Menüs zur Auswahl
des Spielmodus und der Erstellung, Teilnahme an und Durchführung von Turnieren erweitert werden.

Das Spiel soll um ein Strategiemanagement erweitert werden, um Strategien über mehrere Sitzungen
hinweg bearbeiten zu können, und gleichzeitig Zugriff auf mehrere Strategien zu haben, die getrennt
voneinander evaluiert und bearbeitet werden können.

Die Spielfläche muss durch einen zweiten, durch eine nutzerdefinierte Strategie gesteuerten Roboter
mit eigener Lebensanzeige, erweitert werden. Die Interaktion zwischen den Robotern muss so gestaltet
werden, dass unterschiedliche Ansätze zwischen Strategien (z.B. ein Fokus auf viel Bewegung,
effizientes Scannen oder Ausweichen) erfolgsversprechend sein können. Das Geschwindigkeit des
Gameplays sollte so angepasst werden, dass die Interaktion der Roboter und ihre Bewegungen gut
nachvollziehbar sind. Für die Turnierdurchführung muss das Spielfeld außerdem um eine
Benutzeroberfläche zum Abbruch von Kämpfen und eine Anzeige der verbleibenden Zeit und Rundenzahl
erweitert werden. 

Als zusätzliches Ziel soll die Konzeption und Umsetzung eines immersiven, thematisch passenden
Designs gesetzt werden, damit der Fantasy-Aspekt des Spiels und der allgemeine Spielspaß gestärkt
wird.

\chapter{Implementierung}

Aufbau: Alte Architektur (hier auch Anpassug an Roboter GO für Gamplay (scan, kollision, etc)), Neue
Architektur anhand von a)

(Technische Anforderungen)

Die Architektur des CTGameStudio in der Erstversion musste angepasst werden, um die Anforderungen
des neuen Spielmodus zu erfüllen. Das Hinzufügen eines zweiten Roboters erfordert, mehrere Instanzen
der blockly-Komponente und des Strategie-Interpreters zu erzeugen (etwas generischer formulieren).
Die gleichzeitige Ausführung mehrerer Strategien muss ermöglicht werden. Die Gameloop und
Roboterspielobjekte müssen so angepasst werden, dass die Roboter sinnvoll miteinander interagieren
und sich so verhalten, dass sinnvolles Gameplay möglich ist. Zudem muss das Spiel Dialoge mit
Formularelementen erweitert werden, und das Speichern und Laden von Daten über HTTP-Schnittstellen
des Servers unterstützen. (Statemanagement, + HTML-Komponenten/Render-Strategie)


\section{Alte Architektur}

Das ctGameStudio ist ein Webanwendung. Der Server liefert Seiten an den Browser aus und bietet
Login-Management. Unterstützt wird dies durch das auf Node.JS und MongoDB basiertem
KeystoneJS-Framework \footnote{https://keystonejs.com}. Die Darstellung und das dynamische Verhalten
des Spiels basiert auf browser-seitig ausgeführtem HTML und CSS, und Javascript. Neben
Standard-HTML-Inputs wird Phaser \footnote{https://phaser.io} genutzt, um mittels WebGL
Spielelemente zu rendern.

\begin{figure} \label{architektur-alt} \caption{Architektur des ctGameStudio in der Erstversion.
  Phaser-Spielobjekte sind rot hervorgehoben. Durchgezogene Linien zeigen Eltern-Kind beziehungen,
  wobei die Elternkomponente das Kind erstellt. Gestrichlte Linien zwischen Komponenten zeigen
  Abhängigkeiten, so dass eine Komponente die andere referenziert und internenen Zustand der
  Komponente verändert.} \includegraphics{figures/architektur-alt.pdf} \end{figure}

Komponenten sind als  als Javascript-Funktionen, Objekte oder Klassen definiert. Während eine
Strukturierung dadurch erfolgt, dass Komponenten auf mehrere Dateien aufgeteilt sind, liegen die
Komponenten, Instanzen der Komponenten als auch interne Zustandsvariablen in einem globalen
Namensraum. Dadurch ist direkter Zugriff von eigentlich unabhängigen Komponenten aufeinander möglich
und führt zu einer hohen Kopplung zwischen Komponenten.

\subsection{Main}

Die Hauptfunktion wird nach Laden der Webseite aufgerufen und inialisiert das Phaser-Spiel, welches
das Spielmenü und die Roboter-Umgebung enthält, sowie den Strategieeditor.

\subsection{Strategieeditor}

Der Strategieeditor enthält den Blockly-Oberfläche zur Erstellung eines Programms, den
Strategieinterpreter der für die Ausführung der Strategie zuständig ist, und eine Toolbar mit
Steuerungselemente zur Ausführung der Strategien und Aufruf des Blocklexikons.

\subsection{Blockly}

Die Blockly\footnote{https://developers.google.com/blockly/}-Bibliothek stellt einen Editor zur
blockbasierten Programmierung bereit. Zur Konfiguration des Editors werden die Blöcke
definiert, aus denen der Anwender sein Programm zusammenstellen kann. Die Konfiguration besteht aus
einer mit XML strukturierten Beschreibung der Blöcke, sowie aus Javascript-Funktionen, die
beschreiben, welcher Code aus einem Block generiert werden soll. \todo{Codebeispiel?} Die
Blockly-Developer-Tools\footnote{https://blockly-demo.appspot.com/static/demos/blockfactory/index.html}
können genutzt werden, um die Block-Konfiguration zu generieren und zu verwalten.

Die Programme, die mit dem Editor erstellt wurden, können als XML exportiert werden. Analog dazu
kann das Programm in den Editor importiert werden. Zusätzlich kann das Programm in Block-Struktur
andhand der Block-Konfiguration in Javascript-Code umgewandelt werden.

Im ctGameStudio wird der Blockly Editor zur Erstellung von Roboterstrategien genutzt. Bei Laden
eines Storylevels wird die Werkzeugleiste und das bereit stehende Programm dynamisch an die
Anforderungen des Levels angepasst. So sind nur die Blöcke verfügbar, die in dem Level benutzt
werden sollen, und der Editor mit einem leeren oder teils vordefiniertem Programm bestückt.

Bei Ausführung der Roboterstrategie wird eine Javascript-Version des Blockprogramms generiert, und
mit dem Strategieinterpreter ausgeführt.

\subsection{Strategieinterpreter}

Der Strategieinterpreter ist dazu da, den aus Blockly generierten Javascript-Code auszuführen, und
so den Roboter auf dem Spielfeld zu steuern. Er basiert auf dem
JS-Interpreter\footnote{https://neil.fraser.name/software/JS-Interpreter/docs.html} von Neil Fraser,
der eine Sandbox-Umgebung zur sicheren Ausführung von Javscript bereit stellt. Das Sandboxing
garantiert, dass der Code isoliert von der Host-Umgebung, also der Javascript-Umgebung in der das
Spiel läuft, ausgeführt werden kann. Der durch den Anwender erstellten Code kann durch Ausführung
keine Crashes oder Endlosschleifen verursachen.

Neben nativer Javascript-Funktionalität (z.B. Rechenoperationen, Schleifen, und dem Definieren und
Ausführung von Funktionen) gibt der Interpreter die Möglichkeit, Funktionen zu definieren, die aus
der Sandbox heraus aufgerufen werden können. Im ctGameStudio wird dies genutzt, um Befehle zur
Steuerung des Roboters bereit zu stellen. Dazu wird für jede Fähigkeit des Roboters eine Funktion
definiert, die eine zugehörige Methode auf dem Roboter-Spielobjekt ausführt.

Bei Ausführung der Strategie wird der Strategiecode in den Interpreter geladen. Dieser stellt dann
eine Funktion bereit, mit der Code schrittweise ausgeführt werden kann. Eine eigens definierte
Schritt-Funktion ist dafür da, den Code so lange auszuführen, bis jeder Befehl abgearbeitet wurde,
bis der Roboter zerstört wurde, oder bis der Roboter als \enquote{busy} markiert wurde. Letzteres
ist immer dann der Fall, wenn der Roboter einen Befehl ausführt. Die Wiederholung wird durch einen
rekursiven Aufruf der Schritt-Funktion hervorgerufen. Wenn der Interpreter viele Befehle ohne
Unterbrechung abarbeitet, kann dies einen Stack Overflow hervorrufen, was Fehlfunktionen der
Strategieausführung nach sich zieht.

Die Schritt-Funktion wird initial beim Start der Ausführung durch den Nutzer aufgerufen. Daraufhin
wird die Funktion immer nach Abschluss einer Aktion des Roboters oder wenn durch ein Spielereignis
(z.B. Kollision mit dem Rand der Spielwelt) die gerade ausgeführte Aktion des Roboters unterbrochen
wurde.


\section{Anpassung der Architektur}

Das CTGameStudio in der Erstversion besteht aus zwei lose definierten Haupt-Komponenten (Abb. 1).
Das Menü und die Arena sind ein Phaser-Spiel, in dem Untermenüs und die Spielfläche (Arena) als
unterschiedliche Phaser-States implementiert sind. Der Strategieeditor, die Strategieausführung und
die Toolbar zum Starten und Stoppen der Strategien sind in einer blockly-Komponente gebündelt. Die
Interaktion zwischen den Komponenten fand über globale Variablen und Funktionen statt, wodurch

Beim Start des CTGameStudios werden beide Komponenten initialisiert. Zu Beginn eines Spiellevels
wird der Editor vom Arena-state zurückgesetzt, so dass er ausschließlich die Blöcke und die leere
oder zum Teil gefüllte Stratgie enthält, mit denen der Lernende die Strategie für das Level
entwickeln soll.

Bei Hinzufügen, Verändern oder Entfernen von Blöcken wird der Javascript-Code zur Steuerung des
Roboters generiert und in den Strategie-Interpreter geladen.


- Refactoring: Modularisierung und Entkopplung um Wartbarkeit zu erhöhen, neue Features einfacher
einbauen zu können - Verstärkte Modularisierung durch Bündeln von Dingen die zusammen gehören -
Strategie-Editor war gebündelt mit Strategie-Interpreter, der die Roboter-Strategie ausführt, der
API des Roboter-Spielobjekts die vom Stratgie-Interpreter, und User Interface-Steuerung - und
verpacken in nativen Javascript, so dass funktionen und Variablen eines Moduls exportiert und von
anderen Modulen importiert werden kann - 


Vorher: Loses html, Spiel-Menü + Arena in Phaser, blockly-Komponente intermingled mit Ausführung des
Codes, (Key-Terms: Main, RoboInterpreter, RoboProgrammer, Menü, Arena, Statusleiste), starke
Kopplung

Einbinden eines zweiten Roboters

Formulare, Turnierplugin, Design -> HTML-Komponenten (Menüs, Statusbar, Toolbar) Phaser nur noch
Arena Modularisierung Entkopplung über Main

Server bekommt neue Rolle, nicht lediglich Bereitstellen von Login und Ausliefern der Seite, sondern
API und Persistenz

\section{Ausführung der Strategien}

- Gameplay erfordert klare, deterministische Koordination zwischen Strategieinterpreter und Gameloop
-> Interpreter nicht parallel zu Gameloop ausführen sondern aus Gameloop aufgerufen

\section{Zustandsmanagement}

Durch neue Dialoge und Interaktionsmöglichkeiten und Kommunikation mit dem Server hat das Spiel an
Komplexität zugenommen. Um trotzdem einfache Wartbarkeit und Erweiterbarkeit des Spiels zu
gewährleisten, wurde ein System zum Management des Programmzustands eingeführt. Als Zustand des
Programms werden die Parameter bezeichnet, die bestimmen, welche Komponenten gerendert werden und
welche Daten diese erhalten.
% Damit wird bestimmt, wie des Spiel zu einem bestimmten Zeitpunkt aussieht, wie es sich verhält,
% und welche Aktionen vom Nutzer entegegen genommen werden können.

...Walkthrough mit Abbildung.... Der initiale Programmzustand bildet also die Daten ab, die die
Komponenten so konfigurieren, dass diese das erste Menü anzeigen.

Änderungen des Zustands werden durch die Abhandlung von \em{Aktionen} herbeigeführt, die auf Aufruf
von Update-Funktionen mappen. Rendering top down, actions up

Immutability, Main-Methode (Besitz/Kontrolle des States) Die main-Funktion enthält die Referenz auf
den Zustand und gibt ihn an.

Vorteile: - Sichtbarkeit von Änderungen führt zu hoher Debugability - Zustandsänderungen durch pure
Funktionen erhöht Testbarkeit und Argumentation über Veränderungen


\begin{lstlisting}{Ausschnitt aus Zustandsmodell} \end{lstlisting}

\subsection{Zustandsmanagement der Phaser-Komponente}

Das Phaser-Framework gibt einen Update-Render-Loop vor, der sich von Funktionsweise des
Update-Render-Loops des in HTML implementierten Teils unterscheidet. Das Framework ruft
kontiniuierlich eine Update-Funktion auf, in der Spielobjekte hinzugefügt und entfernt werden und
Eigenschaften dieser Spielobjekte verändert werden. Das Framework sorgt für das rendering dieser
Objekte.

Key-differences: Objekte sind veränderlich (mutable), Update wird ständig aufgerufen (da sich auch
ständig etwas ändert)

Integration durch ein Event-Interface, welches wiederrum in Actions resultiert welche den
Haupt-Programmzustand verändern. So können beide Welten die Architektur wählen, die am besten zu
ihrem Anwendungszweck passen.

Abbildung State change der zur Anzeige des Spiels führt, Events während des Spiels (onHit),
onFinished.


\subsection{Datenmodell und Client-Server-Kommunikation}

- Strategien, Turnier, Teilnahmen - RESTful API

\section{Anpassung der Gameloop}

- Kollision - Scanvorgang - Geschwindigkeit der Roboter


\chapter{Fazit}

Ausblick

- ua. Auch sinnvolle Sicht für TTurnierteilnehmer

\include{chapter6}


% Appendix chapters to be put here. They will be enumerated with capital letters 
% if you  did not change the \documentclass options.
\begin{appendix}
%\include{appendix_chapterA}
\end{appendix}
%Ende Anhang

%Bibliography
% We strongly recommend to use bibtex to manage your bibliography. It helps you
% structure your references and helps avoiding missing important data for a correct
% quotation. If you have no other idea jabref (http://jabref.sourceforge.net/)
% might be a good idea (Jave runtime environment needed).
% This style is good to use in german master thesis'. You need to have activated
% \usepackage{bigerm} above.
% For english documents just use apalike.
%\bibliographystyle{geralpha}

\bibliographystyle{apalike}
% to finally announce where your bibliography is stored use
\bibliography{references}
% it is also possible to have several files separated by comma. 
%Bibliographie Angaben mit \bibliography{}

%\printbibliography
\end{document}

%%% Local Variables:
%%% mode: latex
%%% TeX-master: t
%%% End:
